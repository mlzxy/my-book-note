\documentclass[10pt,a4paper]{article}
\usepackage[CJKchecksingle,CJKnumber]{xeCJK}
\setCJKmainfont[BoldFont={Adobe Kaiti Std},
ItalicFont={Adobe Kaiti Std}]{Adobe Song Std}
\setCJKsansfont{Adobe Kaiti Std}
\setCJKmonofont{Adobe Fangsong Std}
\punctstyle{hangmobanjiao}
\usepackage{amsmath}
\usepackage{amsfonts}
\usepackage{float}
\usepackage{amssymb}
\usepackage{makeidx}
\usepackage{graphicx}
\renewcommand{\contentsname}{目录}
\renewcommand{\listfigurename}{插图目录}
\renewcommand{\listtablename}{表格目录}
% \renewcommand{\refname}{参考文献}
\renewcommand{\abstractname}{内容概述}
\renewcommand{\indexname}{索引}
\renewcommand{\tablename}{表}
\renewcommand{\figurename}{图}
\newtheorem{shortcoming}{缺点}
\newtheorem{question}{疑问}
\newtheorem{thought}{想法}
\begin{document}
\title{A Framework for Hand Gesture Recognition Based on Accelerometer and EMG Sensors \newline 读书报告}
\date{\today}
% \author{杨万山,付钟奇,张鑫语\newline}
\maketitle
% \tableofcontents
\paragraph{报告大致内容}
在这里先尝试整体描述一遍论文中所讲的识别字体的方法流程,之后在按照顺序对每个流程写一下自己的理解和想法,最后再写一点感想。

\paragraph{方法流程}
\begin{description}
\item [step0] 最基本的要有ACC和EMG传感器的数据采集,不过感觉每次肯定都要戴在非常相同的位置上呀。
\item [step1] 对EMG所采集到的数据利用\emph{sliding window}来进行分段,确定在哪一段时间用户是在写字。
\item [step2] 提取ACC和EMG在用户活跃的阶段所收集的数据,对数据做一些变换以使数据特性更加明显. \\ 这里对ACC的加速度数据进行了正规化(使每段都为固定长度,因为EMG数据要与ACC数据每段尽量相对应,所以必然有一个需要作出一些修整)。对EMG数据进一步分段,还对每段先经过一个Hamming window来减小不连续性所带来的频谱混叠,并还通过了一个离散AR系统。
\item [step3] 利用决策树的思想来对数据空间进行划分(缩小的空间有利于HMM来找到更准确的答案,否则如此小的数据量,如此大的搜索空间,太容易overfitting了) \newline
\begin{itemize}
\item 先根据手臂是否移动 (dynamic or static)
\item 再根据动作维持时间的长短 (short or long duration)
\item 再利用fuzzy Knn的方法来判别相关动作手的方向
\end{itemize}
\subparagraph{}
其实原文并没有很显示地说明这里要对数据空间做划分,更像是在说利用这几个分类器得到的特征都当做特征向量中的一元,而自己感觉这样很不舒服,因为这些特征本来就揭示了数据中的一些\emph{Hiearchy},当做特征向量中一元很可能被埋没。
\subparagraph{}
而且文中的那三种特征粒度完全不同,前两个粒度大,比如我写这个字是手臂会不会动,时间长短。而第三个粒度小,写一个字时手掌方向可以各种变化,所以倾向认为前两个主要用来划分数据,后一个主要作为HMM的input data stream.(当然前两个同时也可以作为input data vecotr).

\item [step4] 将特征数据输入HMM,不过自己觉得这里的HMM明显和正常的HMM有区别
\begin{enumerate}
\item 这里是有EMG和ACC两列(或多列)数据,所以需要特殊处理(或者是换一个想法)
\item 还有一点,就是正常的HMM每个数据点是1*1维度的,而文章并没有指明这里输入的input stream是个什么类型,让我猜的话应该是方向或者其他什么的,但一定要保证是1*1,数据要多维的话,就增加一个HMM stream
\end{enumerate}
\end{description}

\paragraph{每个步骤的单独理解}
前面大致写下了自己对论文所讲的方法的流程,当然其中也加入了一些自己的想法(当然理解很有可能比较偏差) 下面对每个步骤单独详细地写一下理解和想法。


\subparagraph{数据分段}
首先说一下数据的分段吧,这里面用了非常简单的方法,先摆出公式。
\[
EMG_{avg} = \sum_{c=2}^{N_c}EMG_C(t)
\]
\[
E_{MA}(t) = \dfrac{1}{W} \sum_{i = t-W+1}^{t}EMG_{avg}^2(i)
\]
将多个channel的数据累加,根据一段时间内幅值平方和的大小来判断,显然这个方式是有很多优点的。
\begin{itemize}
\item 十分简单易行,对于很多问题,找到能满足需求的最廉价而不是最\emph{fancy}解法是最好的。
\item 显然可以基本上检测出长度大于W的有效信号,只要W合理,Really get the work done.
\end{itemize}
\begin{thought}
我感觉这样处理虽然肯定能够把EMG信号基本上准确分块,而且这样做有很大好处,它基本将每个字(或英语中的词)的书写分隔开,这样直接对应后面的决策树的使用,而决策树的作用----数据空间分解,显然是十分重要的。
由此可见EMG信号分块是多么的 \emph{essential} . \\

但是文中的方法只对EMG信号做了分段,而利用EMG信号的分段作为基础,来对ACC信号做32点的正规化,自己觉得可以对ACC信号提高重视,也应该做一些分段类的工作

\subparagraph
因为ACC信号直接对应运动轨迹,而拿中文的书写来打比方。
虽然中文字极多极复杂,但书写方法是相对很有限的(都是一横一竖,一撇一捺),如果将ACC信号能够对每个子动作进行分割,感觉上会非常好(其实在文中方法里也很大一部分隐含了这种想法,就是类似KNN中所学习出的方向,就好像用一组基来表示一个字),而显然如果ACC没有分好块,那么很可能会取得差一些的效果。  \newline


(其实感觉已经开始纯瞎想了)而按照这种设想,如果能够利用ACC的分块对应不同书写方法,来模拟笔的运动轨迹,那当然是再好不过了。首先,一些手写识别就是基于轨迹的方法,可以借鉴他们的想法。 而且文中所给方法中的ACC-HMM,相当于把所有方向序列作为一个$1*N$的观测序列,而显然,书写一个汉字,汉字是个方块,如果有一定的轨迹信息,那么就相当于从整体看这个书写过程,应该可以减少一些信息量的损失。
而且那样传入HMM的序列就不是方向序列而是类似于{横,竖,点...},或许这个序列会更靠谱一点。
\end{thought}

不过自己还从来没真正见过用过ACC传感器,而且如果真的可以模拟出部分动作轨迹信息,可能又需要做很多操作方法上的修改,所以还只是一个非常\emph{naive}的想法。\\
\subparagraph{}
不过如果模拟动作的话显然不只是需要ACC信号,而是EMG信号也要参与,因为感觉这个问题的主要信息源是ACC,EMG,还有他们之间的联系,而好像并没有在论文中发现太多明显寻找两者之间联系的方法。
就是HMM也是分开去做的,感觉这很值得思考,不过应该需要很绞尽脑汁和美丽的模型。
\subparagraph{} EMG信号如果能用一组小波基来离散表示,用条件随机场以时间为基准结合两个信号还是有可能的,而且都离散表示也没有幅值问题了,不过对条件随机场忘了不少,一会要再看看。而且这样好像还更加
user-independent了。


\subparagraph{关于EMG信号做的变换}
文中说对信号再做Hamming窗截断,来抑制高频干扰,之后通过一个4阶AR模型。\\ 不过自己略觉得这个4阶的离散系统就是个低通滤波器,用来去噪等等作用。不过这里显然如果不同用户,不同环境,噪声是不同的(把不同用户带来的一些差别也当做噪声无妨).
\begin{thought}
这学期我们刚学完信号统计分析,觉得或许可以用一些
自适应滤波的方法来对信号做处理,比如windrow and Hoff 最小均方自适应滤波方法,自己只是在这学期的EDA实验中用过,比较感兴趣就去查原理了,发现有很多自适应的滤波方法。显然,肯定是各有千秋,
真正实战的时候应该需要比较多的分析和思考。
\end{thought}


\subparagraph{关于决策树}
关于决策树,感觉文中所提出的的思想非常的重要,利用关键的数据特征降低决策空间,感觉是非常高明的举动。很多想法写在前面了。

\subparagraph{关于\emph{Multistream HMM}}
略觉得这里的MHMM方法有一点缺憾(当然主要是这个问题确实比较难)就是它并没有很充分地利用两中信号ACC,EMG之间的联系,只是简单地加了一个权,感觉有点不充分
,而前面已经提到这是重要的信息源头。但自己觉得还是要从序列检测的思想入手,因为毕竟动作是有很强的顺序性的,不能简单地把问题建模为一个输入向量$V1$输出$w=0,1,2,3,4...$的这种问题.

\subparagraph{}
这里还要提一下HMM的训练,自己看到后面的一些experiment result,觉得HMM训练还是用的标号数据,
这样训练相对另一种方法简单多了。


\paragraph{自己的一些其他想法}
\subparagraph{内疚,道歉}
说实在的,心里很内疚,大概期中的时候去找的您,而您给我论文后,我却一直没跟您联系,自己感到非常后悔,就好像辜负了您。
原因主要是这学期选课比较多,期末15号考完的,考试持续了挺长一段时间,真的感觉复习要
疯了,而之前也压力略大,因为不敢再拉下都这么难的课了,还考了次托福,好在这次托考的不错,不用再重刷分了。 
\subparagraph{}
不过自己还是在课余看了一点儿计算机方面的书,开源代码和人文类的公开课。。。等等 \\\\
这里觉得需要解释一下,并不是觉得那些东西在当下阶段比看论文重要,而是他们的事物特征使他们更容易fit into自己的生活节奏,更好理解,人文类公开课
更像像休息,不需要整块的长时间,可以一边吃饭一边看等等。。。 
\subparagraph{}
另外自己还是有点后悔在这学期选课选多了,因为如果有更多时间可以在实验室里多做事情,一样可以学知识,并且实验室经历明显是自己更缺乏的。但毕竟事情都过去了,好坏这东西
谁知道呢,都是\emph{history}了,应当关注眼前的事情,不过大四选课时不能在这样抽风了。\\
\subparagraph{兴趣,计划}
正所谓知错就改,自己感觉对这方面是非常有兴趣的(诚实的说,比上一个实验室看的推荐系统有意思多了),很希望能大四在您那里多做一些事情,还是很喜欢搬砖码代码的。\\
另外很不幸的是这个暑假\textbf{自己的实体}(好别扭)没有办法去实验室,因为6月22日就要去交流了,8月17日才能回来。。。

\subparagraph{瞎想}
这里又是开始白日梦了。。。自己发现一种现象,很多学生包括自己,都会平时写很多字,作业啦,复习总结啦,数学推导啦,一些想法啦等等,而且很多人都认为
用纸写东西比在电脑上打字更加利于思考,尤其是做数学物理的时候,但是很多时候写在纸上本上,没地方放,或比较懒散,过后考完就扔掉了。自己很多的想法,对学科的理解的记录,都扔掉了。
非常可惜,因为说不定有一天又会用到。所以自己觉得或许可以做一只笔,用这个笔写字,它会把数据通过\emph{internet}传到云端,之后云端或是字体识别也好或是怎样,总之以某种格式存储下来你所写下的东西。
到时本子丢了,自己还可以再看到自己的工作,这样可能会非常好,而且笔记什么的,也不怕丢了. \\\\
当然这只是一时设想,还没什么思路,实现需要很多心血和工作。



\paragraph{\emph{The End}}
写到这里快结束了,感觉胡话连篇*\_* ,花您这么多精力来看,真是麻烦您了,多谢您。
\end{document}
